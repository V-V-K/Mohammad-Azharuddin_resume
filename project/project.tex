
\documentclass[letterpaper,11pt]{article}
\newlength{\outerbordwidth}
\pagestyle{empty}
\raggedbottom
\raggedright
\usepackage[svgnames]{xcolor}
\usepackage{framed}
\usepackage{times}
\usepackage{tocloft}
\usepackage{graphicx}
\usepackage{multirow}
\usepackage[utf8]{inputenc}
\usepackage{tabularx} 
\usepackage{hyperref}
\usepackage{pdfpages}
\title{VINIT_NARAYAN_JHA}
 
\setlength{\outerbordwidth}{3pt}  
\definecolor{shadecolor}{gray}{0.75}   
\definecolor{shadecolorB}{gray}{0.93}  
 
\setlength{\evensidemargin}{-0.25in}
\setlength{\headheight}{0in}
\setlength{\headsep}{0in}
\setlength{\oddsidemargin}{-0.25in}
\setlength{\paperheight}{11in}
\setlength{\paperwidth}{8.5in}
\setlength{\tabcolsep}{0in}
\setlength{\textheight}{9.5in}
\setlength{\textwidth}{7in}
\setlength{\topmargin}{-0.3in}
\setlength{\topskip}{0in}
\setlength{\voffset}{0.1in}

\newcommand{\resitem}[1]{\item #1 \vspace{-2pt}}
\newcommand{\resheading}[1]{\vspace{8pt}
  \parbox{\textwidth}{\setlength{\FrameSep}{\outerbordwidth}
    \begin{shaded}
\setlength{\fboxsep}{0pt}\framebox[\textwidth][l]{\setlength{\fboxsep}{4pt}\fcolorbox{shadecolorB}{shadecolorB}{\textbf{\sffamily{\mbox{~}\makebox[6.762in][l]{\large #1} \vphantom{p\^{E}}}}}}
    \end{shaded}
  }\vspace{-5pt}
}
\newcommand{\ressubheading}[4]{
\begin{tabular*}{6.5in}{l@{\cftdotfill{\cftsecdotsep}\extracolsep{\fill}}r}
  \textbf{#1} & #2 \\
  \textit{#3} & \textit{#4} \\
\end{tabular*}\vspace{-6pt}}
 
\begin{document}



%%%%%%%%%%%%%%%%%%%%%%%%%%%%%%
\resheading{Projects}
%%%%%%%%%%%%%%%%%%%%%%%%%%%%%%
 \begin{enumerate}

\item \textbf {Autonomous Crow Robot:}
Made the robot from scratch which works like thirsty crow in the jungle, picking the pebbles(metallic balls) and placing in the the pitcher(specific area on the area).This story is brought into reallity on the screen using augmented Reality Technology. 
\begin{itemize}
\item \textbf{Technology/Tools:} Embedded C, OpenGL, Path planning, OpenCV, XBee communication, Python 
\end{itemize}
\begin{itemize}
\item \textbf{Youtube Link:} https://www.youtube.com/watch?v=DdPGsyRq_lw&feature=youtu.be
\end{itemize}

\item \textbf {Wire Extrusion:}
Involved with the electronics aspect of Extrusion machine. Combined the working of three stepper motor used in the machine on one single microcontroller(Arduino nano) and made a seperate module for this. Contorlled The speed of the motors using the inbuilt timers of Arduino nano. Desined a mechanism for sweeping the wire on the spool(circular thing on which wire is wrapped) uniformly using a switch and interrupt. 

\begin{itemize}
\item \textbf{Technology/Tools:} Embedded C, Arduino IDE
\end{itemize}
\begin{itemize}
\item \textbf{Video link:} https://drive.google.com/open?id=1QfjfdG8ZZFpbYy9xYUHKUDQQ1lk6nkva
\end{itemize}

 
  \item \textbf{Bite Force(Collaboration with GMCH-32 Hospital):} 
Worked on developing a module for the doctors which can measure the force of teeth while the patient bites using the pressure sensor and determining various health, hygine, disease and recovery of the tooth after any surgery. Also I interfaced an OLED on the module only which shows values of pressure from the tooth at the time of bite only and also the doctor can later recollect the values of the patients of which the bites have been taken later on also. This is done using the EEPROM memory of the Arduino nano.
% \end{itemize}
\begin{itemize}
\item \textbf{Technology/Tools:} Arduino IDE
\end{itemize}

\begin{itemize}
\item \textbf{Report link:} https://drive.google.com/open?id=18qXjPWVtBbSYwQGmHe48f0wIXmOOmh3-efq-MZuRBbI
\end{itemize}

 \item \textbf {Password Based Door Lock:}

Worked on Password Based Door Lock System using 8051 Microcontroller using Embedded C in Keil. 
% \end{itemize}
\begin{itemize}
\item \textbf{Technology/Tools:} Keil, Embedded C, 8051 development kit
\end{itemize}

\begin{itemize}
\item \textbf{Link:} https://drive.google.com/open?id=1YzH3cSRm4CVV6TAUyOUlbocSX6CXINPj
\end{itemize}



\item \textbf{Bus on Time(Currently working):}
 Made a device module which fitted on the bus, will keep sending the GPS location to the Google sheet using the nodeMCU. The coordinates send by the bus will now be fetched from the google sheet and then processes by the backend programme which will keep updating the bus location in the database of the programme and when any user wishes to know the position of the bus he will enter the bus no. and the bus cuurent location will be shown to him. 
% \end{itemize}
\begin{itemize}
\item \textbf{Technology/Tools:} Arduino IDE, Javascript, Python, Sheetsu
\end{itemize}

 \end{enumerate}



\end{document}